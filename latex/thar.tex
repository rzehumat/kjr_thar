%% Generated by Sphinx.
\def\sphinxdocclass{report}
\documentclass[letterpaper,10pt,czech]{sphinxmanual}
\ifdefined\pdfpxdimen
   \let\sphinxpxdimen\pdfpxdimen\else\newdimen\sphinxpxdimen
\fi \sphinxpxdimen=.75bp\relax
\ifdefined\pdfimageresolution
    \pdfimageresolution= \numexpr \dimexpr1in\relax/\sphinxpxdimen\relax
\fi
%% let collapsible pdf bookmarks panel have high depth per default
\PassOptionsToPackage{bookmarksdepth=5}{hyperref}

\PassOptionsToPackage{warn}{textcomp}
\usepackage[utf8]{inputenc}
\ifdefined\DeclareUnicodeCharacter
% support both utf8 and utf8x syntaxes
  \ifdefined\DeclareUnicodeCharacterAsOptional
    \def\sphinxDUC#1{\DeclareUnicodeCharacter{"#1}}
  \else
    \let\sphinxDUC\DeclareUnicodeCharacter
  \fi
  \sphinxDUC{00A0}{\nobreakspace}
  \sphinxDUC{2500}{\sphinxunichar{2500}}
  \sphinxDUC{2502}{\sphinxunichar{2502}}
  \sphinxDUC{2514}{\sphinxunichar{2514}}
  \sphinxDUC{251C}{\sphinxunichar{251C}}
  \sphinxDUC{2572}{\textbackslash}
\fi
\usepackage{cmap}
\usepackage[T1]{fontenc}
\usepackage{amsmath,amssymb,amstext}
\usepackage{babel}



\usepackage{tgtermes}
\usepackage{tgheros}
\renewcommand{\ttdefault}{txtt}



\usepackage[Sonny]{fncychap}
\ChNameVar{\Large\normalfont\sffamily}
\ChTitleVar{\Large\normalfont\sffamily}
\usepackage{sphinx}

\fvset{fontsize=auto}
\usepackage{geometry}


% Include hyperref last.
\usepackage{hyperref}
% Fix anchor placement for figures with captions.
\usepackage{hypcap}% it must be loaded after hyperref.
% Set up styles of URL: it should be placed after hyperref.
\urlstyle{same}

\addto\captionsczech{\renewcommand{\contentsname}{PDF verze:}}

\usepackage{sphinxmessages}
\setcounter{tocdepth}{1}



\title{thar}
\date{28.09.2022}
\release{0.1}
\author{Matej Rzehulka}
\newcommand{\sphinxlogo}{\vbox{}}
\renewcommand{\releasename}{Vydání}
\makeindex
\begin{document}

\ifdefined\shorthandoff
  \ifnum\catcode`\=\string=\active\shorthandoff{=}\fi
  \ifnum\catcode`\"=\active\shorthandoff{"}\fi
\fi

\pagestyle{empty}
\sphinxmaketitle
\pagestyle{plain}
\sphinxtableofcontents
\pagestyle{normal}
\phantomsection\label{\detokenize{index::doc}}


\sphinxstepscope


\chapter{PDF verze}
\label{\detokenize{pdf_link:pdf-verze}}\label{\detokenize{pdf_link::doc}}
\sphinxAtStartPar
\DUrole{xref,myst}{PDF verze}

\sphinxstepscope


\chapter{Úvod}
\label{\detokenize{prednaska_19092022:uvod}}\label{\detokenize{prednaska_19092022::doc}}

\section{Obsah a cíle předmětu}
\label{\detokenize{prednaska_19092022:obsah-a-cile-predmetu}}\begin{itemize}
\item {} 
\sphinxAtStartPar
praktická aplikace výpočetních kódů v termohydraulice \textendash{} zejména CFD a subkanálová analýza

\end{itemize}


\section{Teorie k proudění a sdílení tepla (opakování)}
\label{\detokenize{prednaska_19092022:teorie-k-proudeni-a-sdileni-tepla-opakovani}}\begin{itemize}
\item {} 
\sphinxAtStartPar
více info viz THNJ2, 3

\end{itemize}


\subsection{Proudění \textendash{} jednofázové}
\label{\detokenize{prednaska_19092022:proudeni-jednofazove}}\begin{itemize}
\item {} 
\sphinxAtStartPar
chceme: \(p(x, t)\), \(w(x, t)\), \(h(x, t)\) \textendash{}\textgreater{} 5 neznámých

\item {} 
\sphinxAtStartPar
tedy potřebujeme 5 rovnic (níže jsou rovnice kontinuity, Navier\sphinxhyphen{}Stokesovy rovnice \textendash{} ve formulaci pro stlačitelné Newtonovské tekutiny a zákon zachování energie)

\end{itemize}
\begin{equation*}
\begin{split}    \frac{\partial \rho}{\partial \tau} + \frac{\partial (\rho w_i)}{\partial x_i} &= 0 \\
    \frac{\partial (\rho w_i)}{\partial \tau} + \frac{\partial w_i}{\partial x_j}(\rho w_j) &= \rho K_i - \frac{\partial p}{\partial x_i} + \frac{\tau_{ij}}{\partial x_j} \\
    \frac{\partial (\rho h)}{\partial \tau} + \frac{\partial (\rho h w_j)}{\partial x_j} &= \frac{\partial p}{\partial \tau} + w_j \frac{\partial p}{\partial x_j} + \frac{\partial}{\partial x_i}\left(\lambda \frac{\partial T}{\partial x_i}\right) + \tau_{ij} \frac{\partial w_i}{\partial x_j} + K_i \rho w_i + q_V - \frac{\partial q_{r,i}}{\partial x_j}
\end{split}
\end{equation*}
\sphinxAtStartPar
kde
\begin{equation*}
\begin{split}    \frac{\partial \tau_{ij}}{\partial x_j} &= \frac{\partial}{\partial x_j}\left[\eta\left(\frac{\partial w_i}{\partial x_j} + \frac{\partial w_j}{\partial x_i} - \frac{2}{3}\delta_{ij}\frac{\partial w_k}{\partial x_k}\right) + \xi \delta_{ij}\frac{\partial w_k}{\partial x_k}\right] \\\end{split}
\end{equation*}
\sphinxAtStartPar
Dále je potřeba dodat
\begin{itemize}
\item {} 
\sphinxAtStartPar
vztahy pro transportní veličiny
\begin{itemize}
\item {} 
\sphinxAtStartPar
stavové rovnice

\item {} 
\sphinxAtStartPar
další vztahy

\end{itemize}

\item {} 
\sphinxAtStartPar
konstitutivní vztahy \sphinxhyphen{} popisují specifické situace, např.
\begin{itemize}
\item {} 
\sphinxAtStartPar
proudění z povrchu (vztahy s Nu)

\item {} 
\sphinxAtStartPar
…

\end{itemize}

\item {} 
\sphinxAtStartPar
okrajové podmínky
\begin{itemize}
\item {} 
\sphinxAtStartPar
v případě nestacionárních dějů se počáteční stav může určit např. stacionárním výpočtem

\end{itemize}

\end{itemize}


\subsection{Dvoufázové proudění}
\label{\detokenize{prednaska_19092022:dvoufazove-proudeni}}\begin{equation*}
\begin{split}\frac{\partial}{\partial \tau}(\rho_\ell (1-<\alpha>)) + \frac{\partial}{\partial x_i}\left(\rho_\ell w_{\ell,i}(1-<\alpha>)\right) &= \Gamma_\ell \\
\frac{\partial}{\partial \tau}(\rho_v <\alpha>) + \frac{\partial}{\partial x_i}\left(\rho_v w_{v,i}<\alpha>\right) &= \Gamma_v \\
TODO
\end{split}
\end{equation*}\begin{itemize}
\item {} 
\sphinxAtStartPar
je potřeba více vztahů \textendash{} např. konstitutivní vztahy pro výpočet členů \(\Gamma\)

\end{itemize}


\section{Problémy s řešením proudění v praxi}
\label{\detokenize{prednaska_19092022:problemy-s-resenim-proudeni-v-praxi}}\begin{itemize}
\item {} 
\sphinxAtStartPar
výpočetní prostředky jsou omezené
\begin{itemize}
\item {} 
\sphinxAtStartPar
vztahy je potřeba zjednodušovat (k dosažení rozumné výpočetní náročnosti)

\item {} 
\sphinxAtStartPar
typicky se balancuje rychlost výpočtu proti přesnosti (a realističnosti)

\end{itemize}

\item {} 
\sphinxAtStartPar
je potřeba zvolit vhodnou metodu

\end{itemize}


\subsection{Aspekty ovlivňující výběr výpočetní metody}
\label{\detokenize{prednaska_19092022:aspekty-ovlivnujici-vyber-vypocetni-metody}}\begin{itemize}
\item {} 
\sphinxAtStartPar
geometrie problému

\item {} 
\sphinxAtStartPar
probíhající fyzikální jevy
\begin{itemize}
\item {} 
\sphinxAtStartPar
proudění \sphinxhyphen{} jednofázové x vícefázové

\item {} 
\sphinxAtStartPar
fázové přeměny

\item {} 
\sphinxAtStartPar
chemické reakce

\item {} 
\sphinxAtStartPar
rychlost změn \sphinxhyphen{} velikost časového kroku musí být dostatečně malá

\end{itemize}

\item {} 
\sphinxAtStartPar
požadovaná přesnost

\item {} 
\sphinxAtStartPar
nároky na HW a rychlost výpočtu
\begin{itemize}
\item {} 
\sphinxAtStartPar
souvisí s aplikací
\begin{itemize}
\item {} 
\sphinxAtStartPar
můžeme si dovolit čekat (např. bezpečnostní analýzy), nebo ne (on\sphinxhyphen{}line modelování)

\item {} 
\sphinxAtStartPar
počet opakování \sphinxhyphen{} jen 1 výpočet, nebo více (např. při optimalizaci)

\end{itemize}

\end{itemize}

\end{itemize}


\begin{itemize}
\item {} 
\sphinxAtStartPar
předpoklady výpočtu
\begin{itemize}
\item {} 
\sphinxAtStartPar
konzervativní nebo best\sphinxhyphen{}estimate?

\end{itemize}

\end{itemize}


\subsection{Příklady metod \textendash{} výpočet aktivní zóny}
\label{\detokenize{prednaska_19092022:priklady-metod-vypocet-aktivni-zony}}\begin{itemize}
\item {} 
\sphinxAtStartPar
kanálová zóna \textendash{} k příčnému mísení nedochází (RBMK, GCR,…) \sphinxhyphen{} lze aproximovat 1D výpočtem

\item {} 
\sphinxAtStartPar
kazetové palivové soubory s obálkou (např. BWR, VVER\sphinxhyphen{}440,…) \sphinxhyphen{} přetoky mezi PS nejsou, ale chceme uvažovat přenos energie
\begin{itemize}
\item {} 
\sphinxAtStartPar
subkanálová analýza funguje dobře

\end{itemize}

\item {} 
\sphinxAtStartPar
příčné přetoky existují (např. VVER\sphinxhyphen{}1000) \sphinxhyphen{} složitější

\end{itemize}


\section{Metody řešení}
\label{\detokenize{prednaska_19092022:metody-reseni}}\begin{itemize}
\item {} 
\sphinxAtStartPar
CFD
\begin{itemize}
\item {} 
\sphinxAtStartPar
univerzální, široce rozšířené v praxi

\item {} 
\sphinxAtStartPar
nejblíže skutečnému řešení, ale typicky používá zjednodušené modely (např. RANS, LES v modelování turbulence)

\item {} 
\sphinxAtStartPar
velká náročnost na výpočetní prostředky

\item {} 
\sphinxAtStartPar
Př: ANSYS Fluent, Siemens StarCCM+, OpenFOAM,…

\end{itemize}

\item {} 
\sphinxAtStartPar
subkanálová analýza
\begin{itemize}
\item {} 
\sphinxAtStartPar
předpoklad: proudění v axiálním směru je dominantní, ale existují příčné přetoky
\begin{itemize}
\item {} 
\sphinxAtStartPar
rovnice se přepíší do 1D tvaru s členy pro příčné přetoky

\item {} 
\sphinxAtStartPar
„mezi 1D a 2D“

\end{itemize}

\item {} 
\sphinxAtStartPar
potřebuje více konstitutivních vztahů

\item {} 
\sphinxAtStartPar
dnes používané jen ve výpočtech aktivních zón

\item {} 
\sphinxAtStartPar
Př: Cobra, Altham, Subchanflow

\end{itemize}

\item {} 
\sphinxAtStartPar
metoda náhradního média
\begin{itemize}
\item {} 
\sphinxAtStartPar
původní složitá geometrie se nahradí jednodušší „náhradní“ úlohou

\end{itemize}

\item {} 
\sphinxAtStartPar
systémové (integrální) kódy
\begin{itemize}
\item {} 
\sphinxAtStartPar
rovnice jsou integrované přes objem, výpočetní oblast se „nodalizuje“

\item {} 
\sphinxAtStartPar
časté v havarijních analýzách

\item {} 
\sphinxAtStartPar
typicky vhodné i pro velké oblasti jako celá smyčka nebo celý primární okruh

\item {} 
\sphinxAtStartPar
při velkém množství nodů je srovnatelné se subkanálovou analýzou či dokonce s CFD

\item {} 
\sphinxAtStartPar
Př: RELAP, ATHLET, MELCOR, TRACE

\end{itemize}

\item {} 
\sphinxAtStartPar
metoda izolovaného kanálu

\item {} 
\sphinxAtStartPar
code coupling
\begin{itemize}
\item {} 
\sphinxAtStartPar
spojování více různých kódů do složitějších celků

\item {} 
\sphinxAtStartPar
specifické pro daný typ úlohy

\item {} 
\sphinxAtStartPar
časté pro propojení termohydraulických a neutronických výpočtů

\item {} 
\sphinxAtStartPar
Př: RELAP + neutronika, ATHLET + neutronika, DYN3D, TRACE + PARCS, OpenFOAM + Serpent,…

\end{itemize}

\end{itemize}

\sphinxstepscope


\chapter{CFD}
\label{\detokenize{prednaska_23092022:cfd}}\label{\detokenize{prednaska_23092022::doc}}

\section{Principy modelování}
\label{\detokenize{prednaska_23092022:principy-modelovani}}
\sphinxAtStartPar
\sphinxstyleemphasis{CFD = Computational Fluid Dynamics}
\begin{itemize}
\item {} 
\sphinxAtStartPar
řešení diferenciálních rovnic (ZZ hmoty \sphinxhyphen{} rovnice kontinuity, hybnosti \sphinxhyphen{} Navier\sphinxhyphen{}Stokesovy rovnice, energie)

\end{itemize}

\sphinxAtStartPar
Obecně výpočet probíhá ve 2 krocích:
\begin{enumerate}
\sphinxsetlistlabels{\arabic}{enumi}{enumii}{}{.}%
\item {} 
\sphinxAtStartPar
Diskretizace definičního oboru (\(\vec{x}, t\))
\begin{itemize}
\item {} 
\sphinxAtStartPar
různé metody (FVM, FEM, …)

\end{itemize}

\item {} 
\sphinxAtStartPar
Iterační řešení systému algebraických rovnic

\end{enumerate}


\section{Postup}
\label{\detokenize{prednaska_23092022:postup}}\begin{enumerate}
\sphinxsetlistlabels{\arabic}{enumi}{enumii}{}{.}%
\item {} 
\sphinxAtStartPar
Pre\sphinxhyphen{}processing
\begin{itemize}
\item {} 
\sphinxAtStartPar
definice geometrie

\item {} 
\sphinxAtStartPar
generace výpočetní sítě (mesh) \sphinxhyphen{} prostorová, časová

\item {} 
\sphinxAtStartPar
volba modelů (turbulence, fyzikální a chemické procesy,…)
\begin{itemize}
\item {} 
\sphinxAtStartPar
záleží na požadované přesnosti (a dostupném hardwaru)

\item {} 
\sphinxAtStartPar
další faktory \sphinxhyphen{} počt fází, přenos energie,…

\end{itemize}

\item {} 
\sphinxAtStartPar
termofyzikální vlastnosti materiálů (hustota, viskozita,…)
\begin{itemize}
\item {} 
\sphinxAtStartPar
zadané jako konstanty, tabulka hodnot nebo funkce

\end{itemize}

\item {} 
\sphinxAtStartPar
hraniční (popř. i počáteční) podmínky
\begin{itemize}
\item {} 
\sphinxAtStartPar
vstup, výstup, stěny,…

\item {} 
\sphinxAtStartPar
v případě nestacionárních úloh se často používá řešení stacionární úlohy jako počáteční podmínka

\end{itemize}

\end{itemize}

\item {} 
\sphinxAtStartPar
Výpočet
\begin{itemize}
\item {} 
\sphinxAtStartPar
nejsnažší část \sphinxhyphen{} pracuje počítač

\item {} 
\sphinxAtStartPar
vyplatí se sledovat konvergenci \textendash{} např. zkontrolovat, že residua klesají

\end{itemize}

\item {} 
\sphinxAtStartPar
Post\sphinxhyphen{}processing
\begin{itemize}
\item {} 
\sphinxAtStartPar
kontrola, že výsledky dávají smysl

\item {} 
\sphinxAtStartPar
analýza konvergence
\begin{itemize}
\item {} 
\sphinxAtStartPar
zkontrolovat, že další iterace už nemění výsledek

\item {} 
\sphinxAtStartPar
ověřit vliv sítě (meshe) na výsledek
\begin{itemize}
\item {} 
\sphinxAtStartPar
provést výpočet na jemnějším meshi a porovnat výsledky

\end{itemize}

\end{itemize}

\item {} 
\sphinxAtStartPar
zpracování výsledků do požadovaného formátu

\end{itemize}

\end{enumerate}


\section{Chyby}
\label{\detokenize{prednaska_23092022:chyby}}\begin{enumerate}
\sphinxsetlistlabels{\arabic}{enumi}{enumii}{}{.}%
\item {} 
\sphinxAtStartPar
\sphinxstylestrong{Numerické}
\begin{enumerate}
\sphinxsetlistlabels{\arabic}{enumii}{enumiii}{}{.}%
\item {} 
\sphinxAtStartPar
Diskretizační

\end{enumerate}

\sphinxAtStartPar
Souvisí s náhradou derivace diferenčním schématem:
\begin{equation*}
\begin{split}\frac{\partial u}{\partial x} &\approx \frac{u(x+h) - u(x)}{h} + \mathcal{O}(h) \\
\frac{\partial u}{\partial x} &\approx \frac{u(x) - u(x-h)}{h} + \mathcal{O}(h) \\
\frac{\partial u}{\partial x} &\approx \frac{u(x+0.5h) - u(x-0.5h)}{h} + \mathcal{O}(h^2)\end{split}
\end{equation*}
\sphinxAtStartPar
Odhad velikosti diskretizační chyby \textendash{} porovnat výpočty na různých sítích.
\begin{enumerate}
\sphinxsetlistlabels{\arabic}{enumii}{enumiii}{}{.}%
\setcounter{enumii}{1}
\item {} 
\sphinxAtStartPar
Zaokrouhlovací

\sphinxAtStartPar
Souvisí s reprezentací reálných čísel v počítači (\sphinxcode{\sphinxupquote{float}}, \sphinxcode{\sphinxupquote{double}}).

\sphinxAtStartPar
Diskretizační a zaokrouhlovací chyby jdou proti sobě:
\begin{itemize}
\item {} 
\sphinxAtStartPar
snížení diskretizační chyby vede k většímu počtu výpočtů, což může vést k větší zaokrouhlovací chybě (záleží však na typu úlohy, stabilitě schématu,…)

\end{itemize}

\item {} 
\sphinxAtStartPar
Chyba konvergence

\end{enumerate}
\begin{itemize}
\item {} 
\sphinxAtStartPar
výpočet různých veličin (rychlosti, teploty,…) konverguje různě rychle

\item {} 
\sphinxAtStartPar
pro kontrolu stačí sledovat např. velikost residua v závislosti na iteraci

\item {} 
\sphinxAtStartPar
souvisí s iteračním schématem (typem a hyperparametry) pro řešení maticové úlohy

\end{itemize}
\begin{enumerate}
\sphinxsetlistlabels{\arabic}{enumii}{enumiii}{}{.}%
\setcounter{enumii}{3}
\item {} 
\sphinxAtStartPar
Chyba diskretizace fyzikálního modelu

\end{enumerate}
\begin{itemize}
\item {} 
\sphinxAtStartPar
souvisí se zadáním a modelovacími chybami

\item {} 
\sphinxAtStartPar
TODO

\end{itemize}

\item {} 
\sphinxAtStartPar
\sphinxstylestrong{Modelovací}
\begin{enumerate}
\sphinxsetlistlabels{\arabic}{enumii}{enumiii}{}{.}%
\item {} 
\sphinxAtStartPar
Příliš zjednodušená geometrie

\end{enumerate}
\begin{itemize}
\item {} 
\sphinxAtStartPar
např. špatné použití symetrie nebo zanedbání malé oblasti

\end{itemize}
\begin{enumerate}
\sphinxsetlistlabels{\arabic}{enumii}{enumiii}{}{.}%
\setcounter{enumii}{1}
\item {} 
\sphinxAtStartPar
Fyzikální modely

\end{enumerate}
\begin{itemize}
\item {} 
\sphinxAtStartPar
souvisí s modelováním fyzikálních procesů

\item {} 
\sphinxAtStartPar
např. volba modelu turbulence
\begin{itemize}
\item {} 
\sphinxAtStartPar
RANS je rychlé, ale dělá značná zjednodušení

\item {} 
\sphinxAtStartPar
LES řeší některé struktury v proudění, malé struktury pak modeluje; více náročné na HW než RANS, méně než LES

\item {} 
\sphinxAtStartPar
DNS řeší přímo Navier\sphinxhyphen{}Stokesovy rovnice, ale je vysoce náročné na HW

\end{itemize}


\begin{savenotes}\sphinxattablestart
\centering
\begin{tabulary}{\linewidth}[t]{|T|}
\hline
\sphinxstyletheadfamily 
\sphinxAtStartPar
\sphinxincludegraphics{{rans_les_dns}.png}
\\
\hline
\sphinxAtStartPar
Vlevo: Srovnání DNS (a), LES (b) a RANS (c) simulace proudění z trysky (Italian Agency For New Energy Technologies 2006). Vpravo: Schematické znázornění rozdílu mezi RANS, LES and DNS modelováním (Deck et al. 2014). Převzato z ResearchGate, příspěvek uživatele Alwin Hopf, přeloženo. Originál viz https://www.researchgate.net/figure/Left\sphinxhyphen{}Comparison\sphinxhyphen{}of\sphinxhyphen{}a\sphinxhyphen{}DNS\sphinxhyphen{}a\sphinxhyphen{}LES\sphinxhyphen{}b\sphinxhyphen{}and\sphinxhyphen{}RANS\sphinxhyphen{}c\sphinxhyphen{}simulation\sphinxhyphen{}of\sphinxhyphen{}a\sphinxhyphen{}jet\sphinxhyphen{}flow\sphinxhyphen{}Italian\_fig1\_330765625
\\
\hline
\end{tabulary}
\par
\sphinxattableend\end{savenotes}
\begin{itemize}
\item {} 
\sphinxAtStartPar
stlačitelnost tekutin \textendash{} často se aproximují jako nestlačitelné \sphinxhyphen{} ale to není použitelné vždy

\item {} 
\sphinxAtStartPar
potřeba předem odhadnout, co je použitelné (např. \(v \ll c\), kde \(c\) je rychlost zvuku, pak aproximace nestlačitelné tekutiny lze použít)

\end{itemize}

\end{itemize}
\begin{enumerate}
\sphinxsetlistlabels{\arabic}{enumii}{enumiii}{}{.}%
\setcounter{enumii}{2}
\item {} 
\sphinxAtStartPar
Termofyzikální vlastnosti

\end{enumerate}
\begin{itemize}
\item {} 
\sphinxAtStartPar
často aproximujeme konstantami, ale někdy je potřeba poskytnout závislost \(x(\rho)\) nebo \(x(T)\)

\item {} 
\sphinxAtStartPar
např. přirozená konvekce

\end{itemize}
\begin{enumerate}
\sphinxsetlistlabels{\arabic}{enumii}{enumiii}{}{.}%
\setcounter{enumii}{3}
\item {} 
\sphinxAtStartPar
Hraniční podmínky

\item {} 
\sphinxAtStartPar
Počáteční podmínky

\item {} 
\sphinxAtStartPar
Uživatelské funkce

\item {} 
\sphinxAtStartPar
Fázové přechody

\end{enumerate}

\end{enumerate}


\section{Tvorba geometrie}
\label{\detokenize{prednaska_23092022:tvorba-geometrie}}
\sphinxAtStartPar
2 možnosti:
\begin{enumerate}
\sphinxsetlistlabels{\arabic}{enumi}{enumii}{}{.}%
\item {} 
\sphinxAtStartPar
Import modelu z CAD (např. \sphinxcode{\sphinxupquote{stl}} formát)

\end{enumerate}
\begin{itemize}
\item {} 
\sphinxAtStartPar
často zdroj chyb \textendash{} model se nenaimportuje správně,…

\end{itemize}
\begin{enumerate}
\sphinxsetlistlabels{\arabic}{enumi}{enumii}{}{.}%
\setcounter{enumi}{1}
\item {} 
\sphinxAtStartPar
Vlastní SW pro tvorbu geometrie v CFD programu

\end{enumerate}
\begin{itemize}
\item {} 
\sphinxAtStartPar
Odstraňuje problém s importem

\item {} 
\sphinxAtStartPar
Modelovací prostředí nemusí nabízet funkcionalitu jako samostatný CAD software

\item {} 
\sphinxAtStartPar
2 přístupy k implementaci
\begin{itemize}
\item {} 
\sphinxAtStartPar
centrální \sphinxhyphen{} 1 prostředí, kde se pracuje pořád (např. Geostar)

\item {} 
\sphinxAtStartPar
samostatné moduly \sphinxhyphen{} typické u velkých balíků (např. Ansys)

\end{itemize}

\end{itemize}


\subsection{Přístupy ke geometrii}
\label{\detokenize{prednaska_23092022:pristupy-ke-geometrii}}\begin{enumerate}
\sphinxsetlistlabels{\arabic}{enumi}{enumii}{}{.}%
\item {} 
\sphinxAtStartPar
CLI (\sphinxstyleemphasis{command line interface}) + náhled

\end{enumerate}
\begin{itemize}
\item {} 
\sphinxAtStartPar
expert\sphinxhyphen{}friendly

\end{itemize}
\begin{enumerate}
\sphinxsetlistlabels{\arabic}{enumi}{enumii}{}{.}%
\setcounter{enumi}{1}
\item {} 
\sphinxAtStartPar
myš + ikony

\item {} 
\sphinxAtStartPar
od nejmenších modelů
\sphinxhyphen{} křivky \sphinxhyphen{}\textgreater{} plochy \sphinxhyphen{}\textgreater{} objemy
\sphinxhyphen{} nejdřív geometrie, pak síť

\item {} 
\sphinxAtStartPar
od sítě
\sphinxhyphen{} deformace apod.
\sphinxhyphen{} typické pro starší SW

\end{enumerate}


\section{Diskretizace geometrie}
\label{\detokenize{prednaska_23092022:diskretizace-geometrie}}\begin{itemize}
\item {} 
\sphinxAtStartPar
v CFD může mít velký vliv na výsledek (v pevnostních výpočtech spíš menší)

\end{itemize}


\subsection{Prvky}
\label{\detokenize{prednaska_23092022:prvky}}\begin{itemize}
\item {} 
\sphinxAtStartPar
dle dimenzionality:
\begin{itemize}
\item {} 
\sphinxAtStartPar
1D
\begin{itemize}
\item {} 
\sphinxAtStartPar
úsečka / křivka

\item {} 
\sphinxAtStartPar
obsahuje \textgreater{}= 2 nody (2 \sphinxhyphen{} úsečka, 3 \sphinxhyphen{} quadratic curve, 4 \sphinxhyphen{} cubic curve)

\end{itemize}

\item {} 
\sphinxAtStartPar
2D
\begin{itemize}
\item {} 
\sphinxAtStartPar
trojúhelníky (\textgreater{}= 3 nody), čtyřhrany (\textgreater{}= 4 nody)

\item {} 
\sphinxAtStartPar
strana může být křivá (pak potřebuje \textgreater{} 2 nody)

\end{itemize}

\item {} 
\sphinxAtStartPar
3D
\begin{itemize}
\item {} 
\sphinxAtStartPar
hexahedron \sphinxhyphen{} 6 stěn, \textgreater{}= 8 nodů, nemusí být pravoúhlý

\item {} 
\sphinxAtStartPar
hranol (prism)

\item {} 
\sphinxAtStartPar
pyramid

\item {} 
\sphinxAtStartPar
tetrahedron

\item {} 
\sphinxAtStartPar
dodecahedron

\item {} 
\sphinxAtStartPar
polyhedra \sphinxhyphen{} libovolný tvar (mnohostěn)

\end{itemize}

\end{itemize}

\end{itemize}


\subsection{Požadavky na síť}
\label{\detokenize{prednaska_23092022:pozadavky-na-sit}}\begin{itemize}
\item {} 
\sphinxAtStartPar
pokrýt celý objem

\item {} 
\sphinxAtStartPar
prvky se nesmí překrývat
\begin{itemize}
\item {} 
\sphinxAtStartPar
výjimka \sphinxhyphen{} síť typu „chiméra“ \sphinxhyphen{} překryv je úmyslný

\end{itemize}

\end{itemize}


\subsection{Typy sítí}
\label{\detokenize{prednaska_23092022:typy-siti}}\begin{enumerate}
\sphinxsetlistlabels{\arabic}{enumi}{enumii}{}{.}%
\item {} 
\sphinxAtStartPar
Strukturované = pravidelné
\begin{itemize}
\item {} 
\sphinxAtStartPar
typ H \sphinxhyphen{} každý nód obklopen stejným množstvím prvků

\item {} 
\sphinxAtStartPar
typ O \sphinxhyphen{} okolo kruhových těles \sphinxhyphen{} radiální

\item {} 
\sphinxAtStartPar
typ C \sphinxhyphen{} kulaté + ostrý bod \sphinxhyphen{} např. křídla

\item {} 
\sphinxAtStartPar
obecně jednodušší na výpočet \sphinxhyphen{} vede na řešení úlohy s pásovou maticí

\end{itemize}

\item {} 
\sphinxAtStartPar
Nestrukturované = nepravidelné
\begin{itemize}
\item {} 
\sphinxAtStartPar
složitější (horší indexování kvůli nepravidelnostem) \sphinxhyphen{}\textgreater{} náročnější

\item {} 
\sphinxAtStartPar
dnes preferováno \textendash{} dává elementy tam, kde je potřeba lepší rozlišení (oblasti s ostrými gradienty)

\end{itemize}

\end{enumerate}
\begin{itemize}
\item {} 
\sphinxAtStartPar
při použití několika různých sítí je potřeba řešit napojení
\begin{itemize}
\item {} 
\sphinxAtStartPar
konformní \sphinxhyphen{} sítě se napojují

\item {} 
\sphinxAtStartPar
nekonformní \sphinxhyphen{} nody nesousedí

\item {} 
\sphinxAtStartPar
rozhraní 2 sítí může být zdrojem chyb

\end{itemize}

\item {} 
\sphinxAtStartPar
Speciální
\begin{itemize}
\item {} 
\sphinxAtStartPar
rotující mříže \textendash{} pro čerpadla a jiné pohyblivé části

\item {} 
\sphinxAtStartPar
dynamické \textendash{} pohybuje se

\item {} 
\sphinxAtStartPar
dynamic refinement \textendash{} zhušťuje se podle potřeby
\begin{itemize}
\item {} 
\sphinxAtStartPar
problém \textendash{} jak správně zvolit kritérium pro zhuštění

\item {} 
\sphinxAtStartPar
použití např. pro dvoufázové proudění \textendash{} zjemnění sítě poblíž fázového rozhraní

\end{itemize}

\end{itemize}

\end{itemize}


\subsection{Posuzování kvality sítě}
\label{\detokenize{prednaska_23092022:posuzovani-kvality-site}}\begin{enumerate}
\sphinxsetlistlabels{\arabic}{enumi}{enumii}{}{.}%
\item {} 
\sphinxAtStartPar
hustota sítě

\end{enumerate}
\begin{itemize}
\item {} 
\sphinxAtStartPar
balancování HW náročnosti proti přesnosti

\item {} 
\sphinxAtStartPar
doporučuje se provést výpočet na 2 různě hustých sítích a sledovat konvergenci hodnot

\end{itemize}
\begin{enumerate}
\sphinxsetlistlabels{\arabic}{enumi}{enumii}{}{.}%
\setcounter{enumi}{1}
\item {} 
\sphinxAtStartPar
parametry prvků

\item {} 
\sphinxAtStartPar
poměr stran
\sphinxhyphen{} optimálně blízko 1 (tj. prvky nejsou příliš roztáhnuté v 1 dimenzi
\sphinxhyphen{} výjimka \sphinxhyphen{} prismatické prvky pro rozlišení mezní vrstvy

\item {} 
\sphinxAtStartPar
šikmost (skewness) \sphinxhyphen{} zkosení

\item {} 
\sphinxAtStartPar
ortogonalita

\end{enumerate}

\sphinxstepscope


\chapter{Instalace balíku ANSYS Fluent}
\label{\detokenize{instalace_fluentu:instalace-baliku-ansys-fluent}}\label{\detokenize{instalace_fluentu::doc}}


\renewcommand{\indexname}{Rejstřík}
\printindex
\end{document}